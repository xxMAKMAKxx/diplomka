% Soubory musí být v kódování, které je nastaveno v příkazu \usepackage[...]{inputenc}

\documentclass[%
%  draft,    				  % Testovací překlad
  12pt,       				% Velikost základního písma je 12 bodů
  a4paper,    				% Formát papíru je A4
%  oneside,      			% Jednostranný tisk (výchozí)
%% Z následujicich voleb lze použít maximálně jednu:
%	dvipdfm  						% výstup bude zpracován programem 'dvipdfm' do PDF
%	dvips	  						% výstup bude zpracován programem 'dvips' do PS
%	pdftex							% překlad bude proveden programem 'pdftex' do PDF (výchozí)
	unicode,						% Záložky a metainformace budou v kódování unicode
]{report}				    	% Dokument třídy 'zpráva'

\usepackage[utf8]		%	Kódování zdrojových souborů je UTF-8
	{inputenc}					% Balíček pro nastavení kódování zdrojových souborů

\usepackage{graphics}
\usepackage{epstopdf}


\usepackage[				% Nastavení okrajů
	bindingoffset=10mm,		% Hřbet pro vazbu
	hmargin={25mm,25mm},	% Vnitřní a vnější okraj
	vmargin={25mm,34mm},	% Horní a dolní okraj
	footskip=17mm,			% Velikost zápatí
	nohead,					% Bez záhlaví
	marginparsep=2mm,		% Vzdálenost poznámek u okraje
	marginparwidth=18mm,	% Šířka poznámek u okraje
]{geometry}

\usepackage{sectsty}
	%přetypuje nadpisy všech úrovní na bezpatkové, kromě \chapter, která je přenastavena zvlášť v thesis.sty
	\allsectionsfont{\sffamily}

\usepackage{graphicx} % Balíček 'graphicx' pro vkládání obrázků
											% Nutné pro vložení log školy a fakulty

\usepackage[bottom]{footmisc} %foot notes budou na spodu stranky
\usepackage{siunitx}
\usepackage[table,xcdraw]{xcolor} %for tables
\usepackage{multirow}
\usepackage[justification=centering]{caption}
\captionsetup[table]{position=bottom}

\usepackage[
	nohyperlinks				% Nebudou tvořeny hypertextové odkazy do seznamu zkratek
]{acronym}						% Balíček 'acronym' pro sazby zkratek a symbolů
											% Nutné pro použití prostředí 'seznamzkratek' balíčku 'thesis'

\usepackage[
	breaklinks=true,		% Hypertextové odkazy mohou obsahovat zalomení řádku
	hypertexnames=false % Názvy hypertextových odkazů budou tvořeny
											% nezávisle na názvech TeXu
]{hyperref}						% Balíček 'hyperref' pro sazbu hypertextových odkazů
											% Nutné pro použití příkazu 'nastavenipdf' balíčku 'thesis'

\usepackage{pdfpages} % Balíček umožňující vkládat stránky z PDF souborů
                      % Nutné při vkládání titulních listů a zadání přímo
                      % ve formátu PDF z informačního systému

\usepackage{enumitem} % Balíček pro nastavení mezerování v odrážkách
  \setlist{topsep=0pt,partopsep=0pt,noitemsep}

\usepackage{cmap} 		% Balíček cmap zajišťuje, že PDF vytvořené `pdflatexem' je
											% plně "prohledávatelné" a "kopírovatelné"

%\usepackage{upgreek}	% Balíček pro sazbu stojatých řeckých písmem
											%% např. stojaté pí: \uppi
											%% např. stojaté mí: \upmu (použitelné třeba v mikrometrech)
											%% pozor, grafická nekompatibilita s fonty typu Computer Modern!

\usepackage{dirtree}		% sazba adresářové struktury

\usepackage[formats]{listings}	% Balíček pro sazbu zdrojových textů
\lstset{
%	Definice jazyka použitého ve výpisech
%    language=[LaTeX]{TeX},	% LaTeX
%	language={Matlab},		% Matlab
	language={C},           % jazyk C
    basicstyle=\ttfamily,	% definice základního stylu písma
    tabsize=2,			% definice velikosti tabulátoru
    inputencoding=utf8,         % pro soubory uložené v kódování UTF-8
    %inputencoding=cp1250,      % pro soubory uložené ve standardním kódování Windows CP1250
		columns=fixed,  %flexible,
		fontadjust=true %licovani sloupcu
    extendedchars=true,
    literate=%  definice symbolů s diakritikou
    {á}{{\'a}}1
    {č}{{\v{c}}}1
    {ď}{{\v{d}}}1
    {é}{{\'e}}1
    {ě}{{\v{e}}}1
    {í}{{\'i}}1
    {ň}{{\v{n}}}1
    {ó}{{\'o}}1
    {ř}{{\v{r}}}1
    {š}{{\v{s}}}1
    {ť}{{\v{t}}}1
    {ú}{{\'u}}1
    {ů}{{\r{u}}}1
    {ý}{{\'y}}1
    {ž}{{\v{z}}}1
    {Á}{{\'A}}1
    {Č}{{\v{C}}}1
    {Ď}{{\v{D}}}1
    {É}{{\'E}}1
    {Ě}{{\v{E}}}1
    {Í}{{\'I}}1
    {Ň}{{\v{N}}}1
    {Ó}{{\'O}}1
    {Ř}{{\v{R}}}1
    {Š}{{\v{S}}}1
    {Ť}{{\v{T}}}1
    {Ú}{{\'U}}1
    {Ů}{{\r{U}}}1
    {Ý}{{\'Y}}1
    {Ž}{{\v{Z}}}1
}

%%%%%%%%%%%%%%%%%%%%%%%%%%%%%%%%%%%%%%%%%%%%%%%%%%%%%%%%%%%%%%%%%
%%%%%%      Definice informací o dokumentu             %%%%%%%%%%
%%%%%%%%%%%%%%%%%%%%%%%%%%%%%%%%%%%%%%%%%%%%%%%%%%%%%%%%%%%%%%%%%

%% Nastavení jazyka při sazbě.
% Pro sazbu češtiny je použit mezinárodní balíček 'babel', použití
% národního balíčku 'czech', ve spojení s programy 'cslatex' a
% 'pdfcslatex' není od verze 3.0 podporován a nedoporučujeme ho.
\usepackage[
%%Nastavení balíčku babel (!!! pri zmene jazyka je potreba zkompilovat dvakrat !!!)
  %main=czech,english       % originální jazyk je čeština (výchozí), překlad je anglicky
  %main=slovak,english      % originální jazyk je slovenčina, překlad je anglicky
   main=english,czech       % originální jazyk je angličtina, překlad je česky
]{babel}    					% Balíček pro sazbu různojazyčných dokumentů; kompilovat (pdf)latexem!

\usepackage{float} %pridani float options pro figury a tabulky
\usepackage{lmodern}	% vektorové fonty Latin Modern, nástupce půvoních Knuthových Computern Modern fontů
\usepackage{textcomp} % Dodatečné symboly
\usepackage[LGR,T1]{fontenc}  % Kódování fontu -- mj. kvůli správným vzorům pro dělení slov

\usepackage[
%% Z následujících voleb lze použít pouze jednu
  %semestral,					%	sazba zprávy semestrálního projektu (nesází se abstrakty, prohlášení, poděkování)
  %bachelor,					%	sazba bakalářské práce
  diploma,						% sazba diplomové práce
  %treatise,          % sazba pojednání o dizertační práci
  %phd,               % sazba dizertační práce
%% Z následujících voleb lze použít pouze jednu
% left,               % Rovnice a popisky plovoucich objektů budou %zarovnány vlevo
  center,             % Rovnice a popisky plovoucich objektů budou zarovnány na střed (vychozi)
]{thesis}   % Balíček pro sazbu studentských prací
                      % Musí být vložen až jako poslední, aby
                      % ostatní balíčky nepřepisovaly jeho příkazy


%% Jméno a příjmení autora ve tvaru
%  [tituly před jménem]{Křestní}{Příjmení}[tituly za jménem]
\autor[Bc.]{Martin}{Kačmarčík}

\hyphenation{repository}
\hyphenation{PlanetLab}
\hyphenation{o-pe-ra-ting}
\hyphenation{di-stri-bu-tion}
\hyphenation{re-im-ple-men-ta-tion}

\usepackage{url}


%% Pohlaví autora/autorky
% Číselná hodnota: 1...žena, 0...muž
\autorpohlavi{0}

%% Jméno a příjmení vedoucího/školitele včetně titulů
%  [tituly před jménem]{Křestní}{Příjmení}[tituly za jménem]
% Pokud osoba nemá titul za jménem, smažte celý řetězec '[...]'
\vedouci[doc.\ Ing.]{Dan}{Komosný}[Ph.D.]

%% Jméno a příjmení oponenta včetně titulů
%  [tituly před jménem]{Křestní}{Příjmení}[tituly za jménem]
% Pokud nemá titul za jménem, smažte celý řetězec '[...]'
% Uplatní se pouze v prezentaci k obhajobě;
% v případě, že nechcete, aby se na titulním snímku prezentace zobrazoval oponent, pouze příkaz zakomentujte;
% u obhajoby semestrální práce se oponent nezobrazuje
%\oponent[doc.\ Mgr.]{Křestní}{Příjmení}[Ph.D.]

%% Název práce:
%  První parametr je název v originálním jazyce,
%  druhý je překlad v angličtině nebo češtině (pokud je originální jazyk angličtina)
\nazev{Application for monitoring of Linux servers}{Aplikace pro monitorování serverů s operačním systémem Linux}

%% Označení oboru studia
% První parametr je obor v originálním jazyce,
% druhý parametr je překlad v angličtině nebo češtině
\oborstudia{Teleinformatika}{Teleinformatics}

%% Označení ústavu
% První parametr je název ústavu v originálním jazyce,
% druhý parametr je překlad v angličtině nebo češtině
%\ustav{Ústav automatizace a měřicí techniky}{Department of Control and Instrumentation}
%\ustav{Ústav biomedicínského inženýrství}{Department of Biomedical Engineering}
%\ustav{Ústav elektroenergetiky}{Department of Electrical Power Engineering}
%\ustav{Ústav elektrotechnologie}{Department of Electrical and Electronic Technology}
%\ustav{Ústav fyziky}{Department of Physics}
%\ustav{Ústav jazyků}{Department of Foreign Languages}
%\ustav{Ústav matematiky}{Department of Mathematics}
%\ustav{Ústav mikroelektroniky}{Department of Microelectronics}
%\ustav{Ústav radioelektroniky}{Department of Radio Electronics}
%\ustav{Ústav teoretické a experimentální elektrotechniky}{Department of Theoretical and Experimental Electrical Engineering}
\ustav{Ústav telekomunikací}{Department of Telecommunications}
%\ustav{Ústav výkonové elektrotechniky a elektroniky}{Department of Power Electrical and Electronic Engineering}

%% Označení fakulty
% První parametr je název fakulty v originálním jazyce,
% druhý parametr je překlad v angličtině nebo v češtině
%\fakulta{Fakulta architektury}{Faculty of Architecture}
\fakulta{Fakulta elektrotechniky a~komunikačních technologií}{Faculty of Electrical Engineering and~Communication}
%\fakulta{Fakulta chemická}{Faculty of Chemistry}
%\fakulta{Fakulta informačních technologií}{Faculty of Information Technology}
%\fakulta{Fakulta podnikatelská}{Faculty of Business and Management}
%\fakulta{Fakulta stavební}{Faculty of Civil Engineering}
%\fakulta{Fakulta strojního inženýrství}{Faculty of Mechanical Engineering}
%\fakulta{Fakulta výtvarných umění}{Faculty of Fine Arts}

\logofakulta[loga/FEKT_zkratka_barevne_PANTONE_CZ]{loga/UTKO_color_PANTONE_CZ}


%% Rok obhajoby
\rok{2018}
\datum{1.\,1.\,1970} % Datum se uplatní pouze v prezentaci k obhajobě

%% Místo obhajoby
% Na titulních stránkách bude automaticky vysázeno VELKÝMI písmeny
\misto{Brno}

%% Abstrakt
\abstrakt{%
PlanetLab Server Manager is an application helping users to manage their project in the PlanetLab network. To help enable the network projects even further, this Diploma thesis aims to improve the current plbmng applciation by re-writing it fully into Python~3 language, extend existing funcionalities and mianly add support for filtering application by their operational status. Due to application being in Bash, this will require complete re-design and re-implementation of the functions to fully utilize perks of Python 3 language. This thesis covers details of the PlanetLab network and analyse its infrastructure, identify problems in the current application, describes all the improvements made to the application and finally analyse the overall status of PlanetLab network using the newly developed tools for status monitoring. Application source code will be available on GitHUB under MIT licesne and application will be available in the PyPI repositories as well.
}{%
PlanetLab Server Manager je aplikace jejímž účelem je pomoc uživatelům akademické sítě PlanetLab spravovat jejich síťové projekty. Tato diplomová práce si dává za úkol zlepšit aplikaci plbmng a to tak, že se aplikace jazykově unifikuje do Python~3, její stávající funkcionalita se rozšíří o další pomocné funkce a přidá se možnost filtrování serverů na základě jejich aktuálního stavu činnosti. Z důvodu nynější implementace v jazyce Bash, budou tato vylepšení vyžadovat kompletní přestavbu aplikace a její následnou realizaci využívající konstrukce a vlastnosti jazyka Python 3. Tato práce pokrývá popis síťe PlanetLab, analyzuje její infrastrukru, rozpoznává stávající problémy v aplikaci, popisuje a definuje vylepšení do aplikace provedená a nakonec analyzuje aktuální stav sítě pomocí nově přidané funkcionality. Zdrojové kódy aplikace budou dostupné na stránkách GitHUB pod licencí MIT a zároveň aplikace bude dostupná v repozitářích PyPI.
}

%% Klíčová slova
\klicovaslova{%
PlanetLab Network, PlanetLab Server Manager, Linux, Virtualization, Multi-processing, Critical section, Python
}{%
PlanetLab Network, PlanetLab Server Manager, Linux, Virtualization, Multi-processing, Critical section, Python
}

%% Poděkování
\podekovanitext{%
I would like to thank my thesis advisor doc. Ing.~DAN KOMOSNÝ~Ph.D.\ for his guidance, great leadership, time at consultations, patience and factual suggestions how to improve this work.
}%


% Zrušení sazby poděkování projektu SIX, pokud není nutné
\renewcommand\vytvorpodekovaniSIX\relax  % do tohoto souboru doplňte údaje o sobě, druhu práce, názvu...

%%%%%%%%%%%%%%%%%%%%%%%%%%%%%%%%%%%%%%%%%%%%%%%%%%%%%%%%%%%%%%%%%%%%%%%%

%%%%%%%%%%%%%%%%%%%%%%%%%%%%%%%%%%%%%%%%%%%%%%%%%%%%%%%%%%%%%%%%%%%%%%%%
%%%%%%     Nastavení polí ve Vlastnostech dokumentu PDF      %%%%%%%%%%%
%%%%%%%%%%%%%%%%%%%%%%%%%%%%%%%%%%%%%%%%%%%%%%%%%%%%%%%%%%%%%%%%%%%%%%%%
%% Při vloženém balíčku 'hyperref' lze použít příkaz '\nastavenipdf'
\nastavenipdf
%  Nastavení polí je možné provést také ručně příkazem:
%\hypersetup{
%  pdftitle={Název studentské práce},    	% Pole 'Document Title'
%  pdfauthor={Autor studenstké práce},   	% Pole 'Author'
%  pdfsubject={Typ práce}, 						  	% Pole 'Subject'
%  pdfkeywords={Klíčová slova}           	% Pole 'Keywords'
%}
%%%%%%%%%%%%%%%%%%%%%%%%%%%%%%%%%%%%%%%%%%%%%%%%%%%%%%%%%%%%%%%%%%%%%%%

\pdfmapfile{=vafle.map}

%%%%%%%%%%%%%%%%%%%%%%%%%%%%%%%%%%%%%%%%%%%%%%%%%%%%%%%%%%%%%%%%%%%%%%%
%%%%%%%%%%%       Začátek dokumentu               %%%%%%%%%%%%%%%%%%%%%
%%%%%%%%%%%%%%%%%%%%%%%%%%%%%%%%%%%%%%%%%%%%%%%%%%%%%%%%%%%%%%%%%%%%%%%
\begin{document}
\pagestyle{empty} %vypnutí číslování stránek

%% Vložení desek generovaných informačním systémem
\includepdf[pages=1]%
  {pdf/student-desky}% název souboru nesmí obsahovat mezery!
% nebo vytvoření desek z balíčku
%\vytvorobalku
\setcounter{page}{1} %resetovani citace stranek - desky se necisluji

%% Vložení titulního listu generovaného informačním systémem
\includepdf[pages=1]%
  {pdf/student-titulka}% název souboru nesmí obsahovat mezery!
% nebo vytvoření titulní stránky z balíčku
%\vytvortitulku
   
%% Vložení zadání generovaného informačním systémem
\includepdf[pages=1]%
  {pdf/student-zadani}% název souboru nesmí obsahovat mezery!
% nebo lze vytvořit prázdný list příkazem ze šablony
%\stranka{}%
%	{\sffamily\Huge\centering ZDE VLOŽIT LIST ZADÁNÍ}%
%	{\sffamily\centering Z~důvodu správného číslování stránek}

%% Vysázení stránky s abstraktem
\vytvorabstrakt

%% Vysázení stránky s rozšířeným abstraktem
% (týká se pouze bc. a dp. prací psaných v angličtině, viz Směrnice rektora 72/2017)
\cleardoublepage
\noindent
{\large\sffamily\bfseries\MakeUppercase{Rozšířený abstrakt}}
\\
PlanetLab Server Manager (plbmng) je aplikace jejímž účelem je podporovat vývoj a~výzkum síťových projektů, které jsou testovány na akademické síti PlanetLab. Síť PlanetLab je celosvětová akademická síť, která obsahuje přes 1300 serverů, a~která nabízí uživatelům infrastrukturu pro vývoj a~testování projektů. Přístup do sítě PlanetLab je zajištěn za pomocí asociace \texttt{CESNET}. Každý uživatel je přiřazen do skupiny zvané \texttt{slice}, jež definuje servery, které jsou uživateli dostupné. Jednotka serveru se v~terminologie PlanetLab nazývá \texttt{node}. Infrastruktura sítě PlanetLab obsahuje virtuální servery s~operačním systémem Linux. Právě pro zjednodušení práce v~síti PlanetLab vznikla aplikace PlanetLab Server Manager. Tato aplikace umožňuje uživatelům vyhledání serverů v~síti PlanetLab na základě jejich geografické polohy. Dále aplikace umožňuje vzdálený přístup na tyto servery pomocí protokolu ssh, případně jejich vykreslení na mapě.
\paragraph{} Aplikace ve své přechozí verzi obsahuje několik problémů. Největšími problémy jsou programová disparita, chyby v~aplikaci, nepřehledný kód, nelogická struktura, nutnost doinstalovat systémové balíčky po instalaci z~repozitářů PyPI, nutnost lokalizace skriptu, náhlé ukončení činnosti při běhu a jiné. Cílem této diplomové práce je aplikaci programově sjednotit do jazyka Python~3, opravit existující chyby a rozšířit její funkcionalitu například o přidání možnosti vyhledávání serverů na základě jejich aktuálního stavu činnosti. Tato práce řeší existující problémy pomocí nového návrhu programových funkcní, úplného přepisu aplikace do jazyka Python~3.
\paragraph{} Prvním krokem k~vyřešení identifikovaných problémů bylo znovu navrhnout programové funkcne pro plné využití výhod jazyka Python~3. Jednotlivé funkce aplikace, které bylo možné oddělit do samostatných skriptů, byly odděleny do knihoven a napsány tak, aby je bylo možné importovat přímo z~hlavního skriptu. Přínosem tohoto přístup je, že jsou funkce dále využitelné i mimo samotnou aplikaci. Jádro aplikace je také psáno jako knihovna a lze ji importovat do jiných aplikací v~případě potřeby. Výhodou zmíněného přístupu je, že je jádro aplikace inicializováno z~jednoduchého skriptu v~spustitelné složce díky čemuž instalátor PyPI skript automaticky rozpozná a~umístí jej po instalaci do spustitelných systémových složek. Díky tomuto vylepšení není uživatel nucen aplikace po instalaci lokalizovat. Struktura složek aplikace byla vylepsena tak, aby zjednodušila orientaci v~zdrojových kódech aplikace. V~rámci práce byla implementována vylepšení jako například přidání funkcionality pro filtraci serverů na základě aktuálního stavu jejich činnosti. Pro tyto účely byla použita nově přidaná interní databáze. Jádro bylo logicky rozděleno na sekce obsahující grafické rozhraní a sekce obsahující vnitřní logické funkce. V~rámci práce byla provedena další vylepšení do aplikace jako například:
\begin{itemize}
	\item Odstranění limitace výsledků vyhledávání díky použití konstrukcí jazyka Python~3, který dovoluje zjedndoušenou práci s~použitou knihovnou pro uživatelské rozhraní.
	\item Výlepšená struktura samotné aplikace byla přepracována pro usnadnění její údržby.
	\item Aplikace byla vyvíjena za pomocí standardu PEP8, který definuje stylizaci kódu pro aplikace psané v~jazyce Python. 
	\item Díky programové unifikaci byla odstraněna nutnost vykonávat jakékoliv před nebo po instalační kroky.
	\item Vylepšení doznalo také vyplňování přihlašovacích údajů. Dřívější nefunkční pole pro jednotlivé parametry bylo nahrazeno funkčním textovým editorem se snadnou orientací.
	\item Byla přidána zmíněná funkce pro filtrování serverů na základě jejich činnosti a~byla implementována logika pro aktualizaci této databáze pomocí multi-processingu.
	\item Funkcionalita pro zobrazení serverů na mapě nyní umožňuje aplikovat filtry na servery před jejich vykreslením na mapě.
	\item Každý bod na mapě nyní po rozkliknutí obsahuje informaci o~danném serveru.
	\item Byla přidána možnost rychlého přístupu na poslední zobrazený server
	\item V menu je nově možnost zobrazit statistiky o síťi PlanetLab.
	\item Byla přidána podpora operačního systému MacOS.
\end{itemize}
V rámci přepisu aplikace byly opraveny existující chyby. Výsledná vylepšená aplikace je aktualizována, včetně popisků, na příslušném repozitáři PyPI a kód je vystaven pod licencí MIT na portalu GitHUB.\\


%% Vysázení prohlaseni o samostatnosti
\vytvorprohlaseni

%% Vysázení poděkování
\vytvorpodekovani

%% Vysázení poděkování projektu SIX
% ----------- zakomentujte pokud neodpovida realite
\vytvorpodekovaniSIX

%% Vysázení obsahu
\obsah

%% Vysázení seznamu obrázků
%\seznamobrazku

%% Vysázení seznamu tabulek
%\seznamtabulek

%% Vysázení seznamu výpisů
%\lstlistoflistings

\cleardoublepage\pagestyle{plain}   % zapnutí číslování stránek


%% Vložení souboru 'text/uvod.tex' s úvodem
\chapter*{Introduction}
\phantomsection
\addcontentsline{toc}{chapter}{Introduction}

Challenge of developing a network project can become a challenging task. Internet is a huge worldwide network that is present all over the world. PlanetLab Network offers a global research network that enables development of new network services. 

%% Vložení souboru 'text/reseni' s popisem reseni práce
\chapter{Plbmng Tool}
\label{chapter:plbmng}
Plbmng application called \texttt{Data miner for PlanetLab} is available at public PyPi repository\footnote{PyPi page of the plbmng tool: https://pypi.org/project/plbmng/}. The tool allows managing \texttt{PlanetLab} nodes, gathering information about them and pulling the latest data from the \texttt{PlanetLab} API service. Its core is written in Bash and additional modules are written in Python 3 \cite{suba1}. At the moment, it is depended on both Bash and Python modules and it's installation consists of several steps:
\begin{itemize}
	\item Installing the application from PyPi repository or downloading the source codes from GitHub.
	\item Installing additional system packages like dialog,pssh and fping.
	\item Locating installation folder and putting symlink into \texttt{\$PATH} directory.
\end{itemize}
After the first start of the application user is required to fill credentials and \texttt{SSH} public key details to be able to access 	exttt{PlanetLab} API and nodes using the menu option \texttt{Credentials}. All the current options in the menu are \texttt{Search nodes} for retrieving a node from internal database, \texttt{Measure Menu} that allows user to schedule gathering of data about the nodes or run the data gathering now, \texttt{Map Menu} that allow user to generate map showing location of the nodes and mentioned \texttt{Settings}. Menu is created using bash library \texttt{dialog} and can be seen in Figure~\ref{fig:planetlaboldmenu}. 

\begin{figure}[H]
	\centering
	\scalebox{0.4}{\includegraphics{obrazky/planetlabmenuold}}
	\caption{Data miner for PlanetLab menu.}
	\label{fig:planetlaboldmenu}
\end{figure}

asd
\section{Current Tool Funcionality}

\section{Current Tool State}
The first problem of the existing tool is language disparity having half of the functionality in Bash and half of the functionality in Python 3. This makes it difficult to make adjusment to the tool as one needs to study a great amount of scripts that are in several different folders. 
\section{Areas of improvement}
easier installation
description of the menu
\chapter{	exttt{PlanetLab} Network}
\label{chapter:	exttt{PlanetLab}network}

\chapter{Linux and Virtualization}
\label{chapter:Linux}

\chapter{Plbmng Tool Improvements}
\label{chapter:improve}

Teoretické zázemí studentské práce vhodně rozdělené do částí.

(Struktura navržená v~této šabloně je nejhrubší možná, po konzultaci s~vedoucím je vhodné zvolit přiléhavější.)


%% Vložení souboru 'text/vysledky' s popisem vysledků práce
\chapter{Plbmng Tool Improvements}
\label{chapter:improve}
TODO: Add how many servers in planetlab are which distros
TODO: How many servers are responding

%% Vložení souboru 'text/zaver' se závěrem
\chapter{Conclusion}
Share the results.

%% Vložení souboru 'text/literatura' se seznamem literatury
%\include{text/literatura}
%\makeatletter
%\def\@openbib@code{\addcontentsline{toc}{chapter}{Bibliography}}
%\makeatother
\bibliographystyle{czplain}
%\begin{flushleft}
\bibliography{text/literatura_mak} % viz. literatura.bib
%\end{flushleft}


%% Vložení souboru 'text/zkratky' se seznam použitých symbolů, veličin a zkratek
\begin{seznamzkratek}{KolikMista}

	\novazkratka{zkDNS}
		{DNS}
		{Domain Name System}
		
	\novazkratka{zkIP}
		{IP}
		{Internet Protocol}
		
	\novazkratka{zkKVM}
		{KVM}
		{Kernel-based Virtual Machine}
		
	\novazkratka{zkMMU}
		{MMU}
		{Memory Management Unit}
		
	\novazkratka{zkCPU}
		{CPU}
		{Central Processing Unit}
		
	\novazkratka{zkCENTOS}
		{CentOS}
		{Community Enterprise Operating System}


\end{seznamzkratek}


%% Začátek příloh
%\prilohy

%% Vysázení seznamu příloh
%\seznampriloh

%% Vložení souboru 'text/prilohy' s přílohami
%\chapter{Content of DVD}
As a part of this thesis there is a CD which contains following items:
\begin{itemize}
	\item Latex source codes in folder \texttt{latex}.
	\item Text version of the thesis in pdf format.
	\item The application source codes in folder \texttt{plbmng}.
\end{itemize}

\chapter{Manual}
This appendix contains manual for the application implemented in this thesis. It describes how to run the application either using PyPI repositories or using source codes on the attached CD. The application was tested on \texttt{Python version 3.7.3}.
\section{Installation from PyPI repositories and running the application}
To install and run application using PyPI repositories, please do the following:\\
\begin{enumerate}
	\item In your terminal run \texttt{pip3 install plbmng}
	\item Confirm that you want to also install dependencies
	\item Once done simply start the application using command \texttt{plbmng}
\end{enumerate}

\section{Running application from attached CD}
To run application using source codes on the attached CD please do the following:\\
\begin{enumerate}
	\item In your terminal navigate into the CD folder: \texttt{cd \$CD\_FOLDER}
	\item Navigate into source code folder: \texttt{cd plbmng}
	\item Run the application: \texttt{./bin/plbmng}
\end{enumerate}

%% Konec dokumentu
\end{document}
