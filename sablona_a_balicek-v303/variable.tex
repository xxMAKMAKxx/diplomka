%% Nastavení jazyka při sazbě.
% Pro sazbu češtiny je použit mezinárodní balíček 'babel', použití
% národního balíčku 'czech', ve spojení s programy 'cslatex' a
% 'pdfcslatex' není od verze 3.0 podporován a nedoporučujeme ho.
\usepackage[
%%Nastavení balíčku babel (!!! pri zmene jazyka je potreba zkompilovat dvakrat !!!)
  %main=czech,english       % originální jazyk je čeština (výchozí), překlad je anglicky
  %main=slovak,english      % originální jazyk je slovenčina, překlad je anglicky
   main=english,czech       % originální jazyk je angličtina, překlad je česky
]{babel}    					% Balíček pro sazbu různojazyčných dokumentů; kompilovat (pdf)latexem!

\usepackage{float} %pridani float options pro figury a tabulky
\usepackage{lmodern}	% vektorové fonty Latin Modern, nástupce půvoních Knuthových Computern Modern fontů
\usepackage{textcomp} % Dodatečné symboly
\usepackage[LGR,T1]{fontenc}  % Kódování fontu -- mj. kvůli správným vzorům pro dělení slov

\usepackage[
%% Z následujících voleb lze použít pouze jednu
  %semestral,					%	sazba zprávy semestrálního projektu (nesází se abstrakty, prohlášení, poděkování)
  %bachelor,					%	sazba bakalářské práce
  diploma,						% sazba diplomové práce
  %treatise,          % sazba pojednání o dizertační práci
  %phd,               % sazba dizertační práce
%% Z následujících voleb lze použít pouze jednu
% left,               % Rovnice a popisky plovoucich objektů budou %zarovnány vlevo
  center,             % Rovnice a popisky plovoucich objektů budou zarovnány na střed (vychozi)
]{thesis}   % Balíček pro sazbu studentských prací
                      % Musí být vložen až jako poslední, aby
                      % ostatní balíčky nepřepisovaly jeho příkazy


%% Jméno a příjmení autora ve tvaru
%  [tituly před jménem]{Křestní}{Příjmení}[tituly za jménem]
\autor[Bc.]{Martin}{Kačmarčík}

\hyphenation{repository}
\hyphenation{PlanetLab}
\hyphenation{o-pe-ra-ting}
\hyphenation{di-stri-bu-tion}
\hyphenation{re-im-ple-men-ta-tion}

\usepackage{url}


%% Pohlaví autora/autorky
% Číselná hodnota: 1...žena, 0...muž
\autorpohlavi{0}

%% Jméno a příjmení vedoucího/školitele včetně titulů
%  [tituly před jménem]{Křestní}{Příjmení}[tituly za jménem]
% Pokud osoba nemá titul za jménem, smažte celý řetězec '[...]'
\vedouci[doc.\ Ing.]{Dan}{Komosný}[Ph.D.]

%% Jméno a příjmení oponenta včetně titulů
%  [tituly před jménem]{Křestní}{Příjmení}[tituly za jménem]
% Pokud nemá titul za jménem, smažte celý řetězec '[...]'
% Uplatní se pouze v prezentaci k obhajobě;
% v případě, že nechcete, aby se na titulním snímku prezentace zobrazoval oponent, pouze příkaz zakomentujte;
% u obhajoby semestrální práce se oponent nezobrazuje
%\oponent[doc.\ Mgr.]{Křestní}{Příjmení}[Ph.D.]

%% Název práce:
%  První parametr je název v originálním jazyce,
%  druhý je překlad v angličtině nebo češtině (pokud je originální jazyk angličtina)
\nazev{Application for monitoring of Linux servers}{Aplikace pro monitorování serverů s operačním systémem Linux}

%% Označení oboru studia
% První parametr je obor v originálním jazyce,
% druhý parametr je překlad v angličtině nebo češtině
\oborstudia{Teleinformatika}{Teleinformatics}

%% Označení ústavu
% První parametr je název ústavu v originálním jazyce,
% druhý parametr je překlad v angličtině nebo češtině
%\ustav{Ústav automatizace a měřicí techniky}{Department of Control and Instrumentation}
%\ustav{Ústav biomedicínského inženýrství}{Department of Biomedical Engineering}
%\ustav{Ústav elektroenergetiky}{Department of Electrical Power Engineering}
%\ustav{Ústav elektrotechnologie}{Department of Electrical and Electronic Technology}
%\ustav{Ústav fyziky}{Department of Physics}
%\ustav{Ústav jazyků}{Department of Foreign Languages}
%\ustav{Ústav matematiky}{Department of Mathematics}
%\ustav{Ústav mikroelektroniky}{Department of Microelectronics}
%\ustav{Ústav radioelektroniky}{Department of Radio Electronics}
%\ustav{Ústav teoretické a experimentální elektrotechniky}{Department of Theoretical and Experimental Electrical Engineering}
\ustav{Ústav telekomunikací}{Department of Telecommunications}
%\ustav{Ústav výkonové elektrotechniky a elektroniky}{Department of Power Electrical and Electronic Engineering}

%% Označení fakulty
% První parametr je název fakulty v originálním jazyce,
% druhý parametr je překlad v angličtině nebo v češtině
%\fakulta{Fakulta architektury}{Faculty of Architecture}
\fakulta{Fakulta elektrotechniky a~komunikačních technologií}{Faculty of Electrical Engineering and~Communication}
%\fakulta{Fakulta chemická}{Faculty of Chemistry}
%\fakulta{Fakulta informačních technologií}{Faculty of Information Technology}
%\fakulta{Fakulta podnikatelská}{Faculty of Business and Management}
%\fakulta{Fakulta stavební}{Faculty of Civil Engineering}
%\fakulta{Fakulta strojního inženýrství}{Faculty of Mechanical Engineering}
%\fakulta{Fakulta výtvarných umění}{Faculty of Fine Arts}

\logofakulta[loga/FEKT_zkratka_barevne_PANTONE_CZ]{loga/UTKO_color_PANTONE_CZ}


%% Rok obhajoby
\rok{2018}
\datum{1.\,1.\,1970} % Datum se uplatní pouze v prezentaci k obhajobě

%% Místo obhajoby
% Na titulních stránkách bude automaticky vysázeno VELKÝMI písmeny
\misto{Brno}

%% Abstrakt
\abstrakt{%
Abstrakt práce v~originálním jazyce
}{%
Překlad abstraktu v~angličtině (nebo češtině pokud je originální jazyk angličtina)
}

%% Klíčová slova
\klicovaslova{%
Klíčová slova v~originálním jazyce
}{%
Překlad klíčových slov v~angličtině nebo češtině
}

%% Poděkování
\podekovanitext{%
Rád bych poděkoval vedoucímu diplomové práce panu Ing.~XXX YYY, Ph.D.\ za odborné vedení, konzultace, trpělivost a podnětné návrhy k~práci.
}%


% Zrušení sazby poděkování projektu SIX, pokud není nutné
%\renewcommand\vytvorpodekovaniSIX\relax