\chapter*{Introduction}
\phantomsection
\addcontentsline{toc}{chapter}{Introduction}

Developing a network project can become a challenging task. Internet is a huge worldwide network and to properly simulate usage and architecture of the internet requires at least several servers on different locations at minimum. PlanetLab Network offers a global research network that enables development of new network services. The goal of this Diploma thesis is to improve the existing tool, make it easier to use and publish the changes by updating the PyPI repositories. PlanetLab Server Manager is a tool that allows users to get information about nodes in the PlanetLab network and creates an user interface that helps interact with them. The current state of the application can be a barrier for more extensive usage of the application and community driven improvements. Diploma thesis aims to re-write the application into Python; a popular community supported multi-platform object oriented programming language \cite{lutz2013learning}. This thesis extends existing tools developed by Ivan Andrašov \cite{andrasov2} and Filip Šuba \cite{suba1}.\\
The approach for achieving the goals of this thesis consists of using existing Bash functions and re-writing them to Python 3. During this process each functions is evaluated whether the used implementation is correct or not. During re-implementation, several enhancements to the tool will be made. Specifically, the enhancements will include removing search limitation, adding library support, eliminating pre and post installations steps, improving credentials set up, writing functions with support of Windows operating system and several minor bug fixes or improvements. To achieve easier usage of the application, main focus is put on removing system package dependencies and scrapping the necessity to localize the installation folder. To achieve better readability improvements to the implementation of functions, their names and names of menu components are added. Special emphasis is laid on logical code structure and good programming practices to empower later community improvements to the tool.\\
In the Chapter~\ref{chapter:planetlabnetwork} the PlanetLab project will be introduced and characterized. Since PlanetLab infrastructure use Linux as the main operating system on nodes, it will be described in Subsection~\ref{subsection:Linux}. Virtualization is described in Subsection~\ref{subsection:Virtualization} as it is the technology used for provisioning the PlanetLab nodes \cite{planetlababout}. Since this thesis improves already existing tool created by other students, in Chapter~\ref{chapter:plbmng} the tool and summary of previous work is reviewed. In the Chapter~\ref{chapter:improve} the improvements made to the tool will be explained in detail.