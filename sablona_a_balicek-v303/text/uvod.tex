\chapter*{Introduction}
\phantomsection
\addcontentsline{toc}{chapter}{Introduction}

Task of developing a network project can become a challenging task. Internet is a huge worldwide network and to properly simulate the usage and architecture of the internet requires at least several servers on different locations at best. PlanetLab Network offers a global research network that enables development of new network services. The goal of this semestral is to improve existing tool, make it easier to use and publish the changes by updating the PyPi repositories. PlanetLab Server Manager is an existing tool that allows users to get information about nodes in the PlanetLab network and creates an user interface that helps interact with them. The current state of the application, which will be described later, can be a barrier for more extensive usage of the application and community driven improvements. Semestral thesis aims to re-write the application into Python; a popular community supported multi-platform object oriented programming language \cite{lutz2013learning}. This thesis extends existing tools developed by Ivan Andrašov \cite{andrasov2} and Filip Šuba \cite{suba1}.\\
The approach to achieve the goals of this thesis is to take existing Bash functions and re-write them to Python 3. During this process each functions is evaluated whether the used implementation is correct or not. To achieve easier usage of the application, main focus is applied onto removing system package dependencies and scrapping the necessity to localize the installation folder. To achieve better readability improvements to the implementation of functions are added by using best coding practices. Special emphasis is laid on logical structure and good programming practices to empower later community improvements to the tool.\\
Since this thesis uses already existing tool created by previous students, in Chapter~\ref{chapter:plbmng} the tool and summary of previous work is reviewed. In the Chapter~\ref{chapter:planetlabnetwork} the PlanetLab project will be introduced and characterized. As Linux is the main operating system nodes uses, in the Chapter~\ref{chapter:Linux} it will be described along with virtualization as it is the technology used for provisioning the PlanetLab nodes \cite{planetlababout}. In the Chapter~\ref{chapter:improve} the improvements made to the Plbmng tool will be explained.