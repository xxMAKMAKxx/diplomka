\chapter{Conclusion}
The goal of this thesis was to get familiarized with the \texttt{plbmng} tool, re-implement the application into Python 3 and update the PyPi repository. Familiarization with the tool has been a pre-requisite of the other goals. In this thesis, the plbmng tool is, in Chapter~\ref{chapter:plbmng}, reviewed, analyzed and described. This thesis also contains discussion over state of the application and possible improvements. As tool uses PlanetLab network, it was introduced, analyzed and described in Chapter~\ref{chapter:planetlabnetwork}. Since PlanetLab network is primarily using Linux and virtualization, these topics were covered in Chapter~\ref{chapter:Linux}.\\
Re-implementation is described in Chapter~\ref{chapter:improve}. The re-implementation was used to also include improvements to the each functions and their logic. Python usage has enabled various improvements such as removing result limitation, having application available as library, increased readability by having functions logically divided in one file instead of being composed from different files into script saved and run from disk, pre and post installation steps were removed, set credentials function has been improved, functions were written with Windows support in mind whenever possible, few minor bugs were fixed and minor improvements were added. The newly re-implemented tool's behavioral diagram has been described in Section~\ref{section:toolbehavior}.\\
Last goal, which was to update the application's PyPi repository\footnote{The \texttt{plbmng} tool is available at: https://pypi.org/project/plbmng/} with new code and information. The repository was successfully updated with the newest code changing from version \texttt{0.1.10} to version \texttt{0.2.1}. The description has been updated containing latest information regarding installation process of the tool. Dependencies were removed to reflect this change. With the update all the goals that this Semestral thesis aimed to accomplish were successfully achieved. Overall, all these changes were prerequisite for further development of the tool and increasing its usability for the PlanetLab users.
\paragraph{Future Development}
As mentioned in this Chapter introduction, the changes done in this Semestral thesis are prerequisite for the future work which will be done in the following Diploma thesis. The Diploma thesis will aim to document use cases of the application and provide documentation using \zk{zkUML} (\zkratkatext{zkUML}) diagrams. Based on this documentation, the tool functionality  will be revised and improved to reflect the use cases. Various existing functions will be improved such as map generation currently doesn't provide much information. Data to the map nodes, like IP address and country can be added. Also, as the tool was developed in two thesis, there are still Python scripts in different folders. This structure will be reviewed and improved to match one solid library to use. Windows support will be further investigated and if possible, various functions will be ported to support both Linux-based and Windows operating systems. Another possible improvement is to add command line interface support so the script can be usable outside the GUI. Work on this improvement has been already started by adding \texttt{crontab} mode to the script which starts monitoring of the available servers. 