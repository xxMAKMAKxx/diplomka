\chapter{Conclusion}
The goal of this Master thesis was to get familiarized with the \texttt{plbmng} tool, re-implement the application into Python 3, extend its functionality by adding an ability to filter the servers by their status and update the PyPI repository of the application. Getting familiarized with the tool has been a pre-requisite of the other goals. As tool is designed to make usage of PlanetLab network easier, it was introduced and described in Chapter~\ref{chapter:planetlabnetwork}. The previous \texttt{plbmng} tool is, in Chapter~\ref{chapter:plbmng}, reviewed and analyzed. This thesis also contains discussion over state of the application and possible improvements. Since PlanetLab network is primarily using Linux and virtualization, these topics were covered in Subsection~\ref{subsection:Linux} and Subsection~\ref{subsection:Virtualization} respectively.\\
The re-implementation itself is described in Chapter~\ref{chapter:improve}. First, application logic was re-defined to use Python 3 strength and its diagram can be seen in Section~\ref{section:redesign}. Main features are \texttt{SQLite3} database where all data are stored and independent library modules providing functionality to the core engine. The re-implementation was used to make improvement to each function. Python usage has enabled various advancements, such as removing result limitation, removing pre and po installation steps, having application available as library, increased readability by having core functions logically divided in one file instead of being composed from different files into script saved and run from disk, pre and post installation steps were removed, set credentials function has been improved, functions were written in a multi-platform way, few minor bugs were fixed and minor improvements were added. Folder structure was completely re-designed to be more transparent and easily oriented. Filter function was added to help find only available nodes. Logic to update availability database in multi-processing way was implemented. Small features like accessing last server or having statistics available in the application are improving plbmng usability. The newly re-implemented tool's behavioral diagram has been described in Section~\ref{section:currentapp}.\\
Last goal, which was to update the application's PyPI repository\footnote{The \texttt{plbmng} tool is available at: https://pypi.org/project/plbmng/} with new code and updated description to fit the latest information. The repository was successfully updated with the newest code changing from version \texttt{0.1.10} to version \texttt{0.3.5}. The description has been updated containing latest information regarding installation process of the tool. Dependencies were removed to reflect this change. With the update all the goals that this Diploma thesis aimed to accomplish were successfully achieved. Overall, described changes will help further development of the tool and increasing its usability for the PlanetLab users.
\paragraph{Future Development}
PlanetLab Server Manager is a useful tool for managing network projects in the PlanetLab network. However, there are still many features that would help to make it even better. Application has great room for improvement in several ways like analyzing and later adding full support of Windows platform, improving gathering of nodes which is limited by \texttt{Open StreetMaps API} which can for example re-use database, added in this thesis, to pull cached information about nodes geological position. This change could improve the node list database update by up to 80\% which is an considerable improvement. Another logic that the application could use is multi-distribution of files. Menu where user could select servers and provide path to its project could save many times to them. Since application codes are open source and available at \texttt{https://github.com/xxMAKMAKxx/plbmng} under MIT license, any contributor willing to help with the project development is most welcomed.