\chapter{Conclusion}
The goal of this Master thesis was to get familiarized with the PlanetLab Server Manager, re-implement the application into Python 3, extend its functionality by adding an ability to filter the servers by their status and update the PyPI repository of the application. Getting familiarized with the tool has been a pre-requisite of the other goals. Planet:Lab Server Manager was described in Chapter~\ref{chapter:planetlabnetwork}. The previous PlanetLab Server Manager tool is, in Chapter~\ref{chapter:plbmng}, reviewed, analyzed and weak points were found. This thesis also contains discussion over state of the application and possible improvements. Since the PlanetLab network is primarily using Linux and virtualization, these technologies were also covered in Subsection~\ref{subsection:Linux} and Subsection~\ref{subsection:Virtualization} respectively.\\
The improvements, as the result of this thesis, are described in Chapter~\ref{chapter:improve}. First, application logic was re-defined to use Python 3 advantags and its diagram can be seen in Section~\ref{section:redesign}. One of the main improvements is \texttt{SQLite3} database where all data are stored and independent library modules providing functionality to the core engine. Python usage has enabled various advancements, such as removing result limitation, removing pre and po installation steps, having application available as library, increased readability by having core functions logically divided in one file instead of being composed from different files into script saved and run from disk, pre and post installation steps were removed, set credentials function has been improved, functions were written in a multi-platform way, few minor bugs were fixed and minor improvements were added. Folder structure was completely re-designed to be more transparent and easily oriented. Filter function was added to help find only available nodes. Logic to update availability database in multi-processing way was implemented. Small features like accessing last server or having statistics available in the application are improving PlanetLab Server Manager usability. The behavioral diagram with the improvements has been described in Section~\ref{section:currentapp}.\\
Last goal was to update the application's PyPI repository\footnote{The PlanetLab Server Manager tool is available at: https://pypi.org/project/plbmng/} with new code and updated description to fit the latest information. The repository was successfully updated with the newest code changing from version \texttt{0.1.10} to version \texttt{0.3.7}. The description has been updated containing latest information regarding installation process of the tool. Dependencies were removed to reflect this change. With the update all the goals that this Diploma thesis aimed to accomplish were successfully achieved. Overall, described changes will help further development of the tool and increasing its usability for the PlanetLab users.