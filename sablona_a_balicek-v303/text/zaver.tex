\chapter{Conclusion}
Goal of this Master thesis was to improve the current of state of the PlanetLab Server Manager (plbmng). PlanetLab Server Manager supports research and development of distributed network services. Application was described in Chapter~\ref{chapter:planetlabnetwork}. In Chapter~\ref{chapter:plbmng}, the previous PlanetLab Server Manager tool is reviewed, analyzed and weak points like program disparity, numerous bugs, result limitation, pre and post installation steps and others were identified and possible improvements were suggested. Since the PlanetLab network is primarily using Linux and virtualization, these technologies were covered in Subsection~\ref{subsection:Linux} and Subsection~\ref{subsection:Virtualization} respectively.\\
The improvements, as the result of this thesis, are described in Chapter~\ref{chapter:improve}. First, application logic was re-defined to use Python 3 advantages and new architecture diagram can be seen in Section~\ref{section:redesign}. One of the main improvements is fast \texttt{SQLite3} database where all data are stored and independent library modules providing functionality to the core engine. Python usage has enabled various advancements, such as removing result limitation, removing pre and po installation steps, having application available as library, increased readability by having core functions logically divided in one file instead of being composed from different files into script saved and run from disk, set credentials function has been improved, functions were written in a multi-platform way, few minor bugs were fixed and minor improvements were added. Folder structure was completely re-designed to be more transparent and easily oriented. Filter function was added to help find only available nodes. Logic to update availability database in multi-processing way was implemented. Minor but useful features features like accessing last server or having statistics available in the application are improving PlanetLab Server Manager usability. Full behavioral diagram with the new improvements is available in Section~\ref{section:currentapp}.\\
Last goal was to update the application at the PyPI repository\footnote{The PlanetLab Server Manager tool is available at: https://pypi.org/project/plbmng/} with a new code and updated description to fit the latest information. The repository was successfully updated with the newest code increment version \texttt{0.1.10} to version \texttt{0.3.7}. The description has been updated containing latest information regarding installation process of the tool. Dependencies were removed to reflect this change. Overall, described changes might help further development of the tool and increase its usability for the PlanetLab users helping develop new distributed network services.